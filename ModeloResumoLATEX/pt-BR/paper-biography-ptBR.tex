
\section*{DADOS BIOGRÁFICOS}

%\balance

\begin{biografia}{Fulano de Tal}
	nascido em 30/02/1960 em Talópoli, é engenheiro eletricista (1983), mestre (1985) e doutor em Engenharia Elétrica (1990) pela Universidade de Tallin.
	
	Ele foi, de 1990 a 1995, coordenador do Laboratório de Tal. Atualmente é professor titular da Universidade de Tal. Suas áreas de interesse são: eletrônica de potência, qualidade do processamento da energia elétrica, sistemas de controle eletrônicos e acionamentos de máquinas elétricas.
	
	Dr. Tal é membro da SOBRAEP, da SBA e do IEEE. Durante o período de 1998 a 2000 foi editor da revista Eletrônica de Potência da SOBRAEP.
\end{biografia}



\begin{biografia}{Fulano de Tal}
	nascido em 30/02/1960 em Talópoli, é engenheiro eletricista (1983), mestre (1985) e doutor em Engenharia Elétrica (1990) pela Universidade de Tallin.
	
	Ele foi, de 1990 a 1995, coordenador do Laboratório de Tal. Atualmente é professor titular da Universidade de Tal. Suas áreas de interesse são: eletrônica de potência, qualidade do processamento da energia elétrica, sistemas de controle eletrônicos e acionamentos de máquinas elétricas.
	
	Dr. Tal é membro da SOBRAEP, da SBA e do IEEE. Durante o período de 1998 a 2000 foi editor da revista Eletrônica de Potência da SOBRAEP.
\end{biografia}


\begin{biografia}{Fulano de Tal}
	nascido em 30/02/1960 em Talópoli, é engenheiro eletricista (1983), mestre (1985) e doutor em Engenharia Elétrica (1990) pela Universidade de Tallin.
	
	Ele foi, de 1990 a 1995, coordenador do Laboratório de Tal. Atualmente é professor titular da Universidade de Tal. Suas áreas de interesse são: eletrônica de potência, qualidade do processamento da energia elétrica, sistemas de controle eletrônicos e acionamentos de máquinas elétricas.
	
	Dr. Tal é membro da SOBRAEP, da SBA e do IEEE. Durante o período de 1998 a 2000 foi editor da revista Eletrônica de Potência da SOBRAEP.
\end{biografia}



\begin{biografia}{Fulano de Tal}
	nascido em 30/02/1960 em Talópoli, é engenheiro eletricista (1983), mestre (1985) e doutor em Engenharia Elétrica (1990) pela Universidade de Tallin.
	
	Ele foi, de 1990 a 1995, coordenador do Laboratório de Tal. Atualmente é professor titular da Universidade de Tal. Suas áreas de interesse são: eletrônica de potência, qualidade do processamento da energia elétrica, sistemas de controle eletrônicos e acionamentos de máquinas elétricas.
	
	Dr. Tal é membro da SOBRAEP, da SBA e do IEEE. Durante o período de 1998 a 2000 foi editor da revista Eletrônica de Potência da SOBRAEP.
\end{biografia}